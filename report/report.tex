\documentclass[11pt]{article}

\usepackage[a4paper, total={16cm, 24cm}]{geometry}
\usepackage[portuguese]{babel}
\usepackage[utf8]{inputenc}
\usepackage{graphicx}
\usepackage{amsmath}
\usepackage{tikz}
    \usetikzlibrary{shadows}
\usepackage{booktabs}
\usepackage[colorlinks=true]{hyperref}
\usepackage{listings}
    \renewcommand\lstlistingname{Listagem}
    \lstset{numbers=left, numberstyle=\tiny, numbersep=5pt, basicstyle=\footnotesize\ttfamily, frame=tb,rulesepcolor=\color{gray}, breaklines=true}
\usepackage{blindtext}

% -------------------------------------------------------------------------------------------
\title
{
    \includegraphics[width=0.4\textwidth]{imgs/university.png}
    \\[0.1cm]
    \textbf{3º Trabalho} \\
    Inteligência Artificial
}

\author
{
    \textbf{Professora:} Irene Rodrigues \\
    \textbf{Realizado por:} Filipe Alfaiate (43315), Miguel de Carvalho (43108), João Pereira (42864) 
}
\date{\today}

% -------------------------------------------------------------------------------------------
%                                Body                                                       %
% -------------------------------------------------------------------------------------------

\begin{document}
\maketitle

% -------------------------------------------------------------------------------------------
\section{}

\hspace{0,6cm}a) \verb|estado_inicial(e(1,  2,  5,  7)).|

b) \verb|terminal(e(0, 0, 0, 0)).|

c) \verb|valor(E, -1, P):- terminal(E), R is P mod 2, R = 1.|

\hspace{0,45cm}\verb|valor(E, 1, P):- terminal(E), R is P mod 2, R=0.|

d) Dado o estado inicial \verb|e(1, 2, 2, 5)| vamos remover \textbf{na linha 4, 4 paus}
(\verb|ret(4,4)|), obtendo um estado \verb|e(1, 2, 2, 1)|.

e) A pesquisa alfa-beta, também conhecida como \textbf{Poda da Árvore} (Pruning) é uma técnica
de compressão de dados que reduz o tamanho da \textbf{árvore do min-max} através da remoção de
secções da árvore que não são críticas e redundantes para a decisão, ou seja, \textbf{reduz a complexidade 
temporal e espacial}.

f)

g) Dado o estado inicial \verb|e(1, 2, 2, 5)| o jogo seria o seguinte:
\newline
\begin{lstlisting}
    ret(4,4)
    e(1,2,2,1)
    1.
    1.
    e(0,2,2,1)
    ret(4,1)
    e(0,2,2,0)
    2.
    1.
    e(0,1,2,0)
    ret(3,2)
    e(0,1,0,0)
    2.
    1.
    e(0,0,0,0)
    pc ganhou
\end{lstlisting}

\newpage

h) Usando \textbf{mini-max}:

\hspace{0,45cm}Dado o estado inicial \verb|e(0, 2, 1, 0)| iria expandir 3 nós.

\hspace{0,45cm}Dado o estado inicial \verb|e(0, 2, 2, 0)| iria expandir 4 nós.

\hspace{0,45cm}Dado o estado inicial \verb|e(0, 3, 2, 0)| iria expandir 5 nós.

\hspace{0,45cm}Dado o estado inicial \verb|e(0, 3, 3, 0)| iria expandir 6 nós.

\hspace{0,45cm}Dado o estado inicial \verb|e(0, 4, 3, 0)| iria expandir 7 nós.

\hspace{0,45cm}Dado o estado inicial \verb|e(0, 4, 4, 0)| iria expandir 8 nós.

\hspace{0,45cm}Dado o estado inicial \verb|e(0, 5, 4, 0)| iria expandir 9 nós.

\hspace{0,45cm}Dado o estado inicial \verb|e(0, 5, 5, 0)| iria expandir 10 nós.

\hspace{0,45cm}Dado o estado inicial \verb|e(0, 6, 5, 0)| iria expandir 11 nós.

\hspace{0,45cm}Dado o estado inicial \verb|e(0, 6, 6, 0)| iria expandir 12 nós.

% -------------------------------------------------------------------------------------------
\section{}

\hspace{0,6cm}a) \verb|estado_inicial(e(p1(0, 0), p2(2, 2), p1)).|

b) \verb|terminal(E):- +op1(E, , ).|

c) \verb|valor(E, -1, P):- terminal(E), R is P mod 2, R = 1.|

\hspace{0,45cm}\verb|valor(E, 1, P):- terminal(E), R is P mod 2, R=0.|


% -------------------------------------------------------------------------------------------
\end{document}